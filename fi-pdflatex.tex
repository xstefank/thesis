\documentclass[oneside,
  digital, %% This option enables the default options for the
           %% digital version of a document. Replace with `printed`
           %% to enable the default options for the printed version
           %% of a document.
  table,   %% Causes the coloring of tables. Replace with `notable`
           %% to restore plain tables.
  nolof,     %% Prints the List of Figures. Replace with `nolof` to
           %% hide the List of Figures.
  nolot,     %% Prints the List of Tables. Replace with `nolot` to
           %% hide the List of Tables.
  %% More options are listed in the user guide at
  %% <http://mirrors.ctan.org/macros/latex/contrib/fithesis/guide/mu/fi.pdf>.
]{fithesis3}
%% The following section sets up the locales used in the thesis.
\usepackage[resetfonts]{cmap} %% We need to load the T2A font encoding
\usepackage[T1,T2A]{fontenc}  %% to use the Cyrillic fonts with Russian texts.
\usepackage[
  main=english, %% By using `czech` or `slovak` as the main locale
                %% instead of `english`, you can typeset the thesis
                %% in either Czech or Slovak, respectively.
  german, russian, czech, slovak %% The additional keys allow
]{babel}        %% foreign texts to be typeset as follows:
%%
%%   \begin{otherlanguage}{german}  ... \end{otherlanguage}
%%   \begin{otherlanguage}{russian} ... \end{otherlanguage}
%%   \begin{otherlanguage}{czech}   ... \end{otherlanguage}
%%   \begin{otherlanguage}{slovak}  ... \end{otherlanguage}
%%
%% For non-Latin scripts, it may be necessary to load additional
%% fonts:
\usepackage{paratype}
\def\textrussian#1{{\usefont{T2A}{PTSerif-TLF}{m}{rm}#1}}
%%
%% The following section sets up the metadata of the thesis.
\thesissetup{
    university    = mu,
    faculty       = fi,
    type          = mgr,
    author        = Martin Štefanko,
    gender        = m,
    advisor       = {Bruno Rossi, PhD},
    title         = {Use of Transactions within a Reactive Microservices Environment},
    TeXtitle      = {Use of Transactions within a Reactive Microservices Environment},
    keywords      = {transactions, Narayana, JTA, reactive, microservices, asynchronous, saga},
    TeXkeywords   = {transactions, Narayana, JTA, reactive, microservices, asynchronous, saga},
}
\thesislong{abstract}{
abstract
}
\thesislong{thanks}{
   thanks
}
%% The following section sets up the bibliography.
\usepackage{booktabs}

\usepackage{makeidx}      %% The `makeidx` package contains
\makeindex                %% helper commands for index typesetting.
%% These additional packages are used within the document:
\usepackage{paralist}
\usepackage{amsmath}
\usepackage{amsthm}
\usepackage{amsfonts}
\usepackage{url}
\usepackage{menukeys}
\begin{document}
\chapter{Introduction}

\chapter{Distributed transaction management}

This chapter introduces the basic concepts of distributed transactions and common problems of managing transactions across multiple nodes.

\section{Consensus protocols}

\subsection{ACID}

\subsection{2PCP}

\section{Saga pattern}

A saga, as described in the original publication \cite{sagas_publ}, is a long lived transaction that can be written as a sequence of transactions that can be interleaved with other transactions. Each operation that is a part of the saga represents an unit of work that can be undone by the compensation action. The saga guarantees that either all operations complete successfully, or the corresponding compensation actions are run for all executed operations to cancel the partial processing.

\subsection{Subtransaction}

\chapter{Communication forms}

\section{CQRS}

\section{Axon framework}

\chapter{Conclusion}



\makeatletter\thesis@blocks@clear\makeatother
\phantomsection %% Print the index and insert it into the
\addcontentsline{toc}{chapter}{Bibliography} %% table of contents.
\printindex

\bibliographystyle{IEEEtran}
\bibliography{IEEEabrv,references}

\appendix %% Start the appendices.

\end{document}
